\chapter{Introduction}
\label{chap:partI_intro}

% Deep Learning models, and in particular Convolutional Neural Networks (CNNs) \cite{jimenez, greenspan}, have shown the ability to outperform the state-of-the-art in many computer vision applications in the past decade. 
One of the reasons for the increasing popularity of data science is the extensive list of successes achieved thanks to statistical approaches capable of learning from data.
Deep Learning is one of such techniques and comprehends a family of models derived from classical Artificial Neural Networks \cite{Rosenblatt1957}. Their distinctive characteristic is the exploitation of efficient implementations available in modern computers to build architectures with many hidden layers, whence the `deep' connotation.
Among these models, Convolutional Neural Networks (CNNs) \cite{jimenez, greenspan} were the first apparent evidence of Deep Learning's potential, showing the ability to outperform the state-of-the-art in many computer vision applications in the past decade. 
Successful examples range from classification and detection of basically any kind of objects \cite{AlexNet, YOLO} to generative models for image reconstruction \cite{reconstruction} and super-resolution \cite{super-resolution}.
Thus, researchers from both academy and industry have started to explore adopting these techniques in fields such as medical imaging and bioinformatics, where the potential impact is vast.
For instance, CNNs have been employed for the identification and localization of tumors \cite{brain_tumor,breast_cancer, ciresan2012deep, cirecsan2013mitosis}, as well as detection of other structures like lung nodules \cite{lung_nodules, meraj2020lung, su2021lung}, skin and breast cancer, diabetic foot \cite{TL_medical_imaging}, colon-rectal polyps \cite{korbar} and more, showing great potential in detecting and classifying biological features \cite{lundervold, sahiner, yadav}.

In the wake of this line of applied research, \Cref{partI} tackles the problem of counting cells into fluorescent microscopy pictures.
In particular, our work aims at developing a Deep Learning approach for automatic recognition and counting of neuronal cells.
We start by exploiting formerly available fluorescence pictures acquired during past experiments, and we collect detailed pixel-wise segmentation maps tracking the location of the cells in such pictures.
Then, we attempt different approaches based on convolutional neural networks trained from scratch in a supervised manner.
In the end, the results are assessed both in terms of cell detection and counting performances. Also, a thorough qualitative assessment is conducted with the help of domain experts.
Importantly, the collected dataset and the best pre-trained model are released to foster future research in this and related areas. 

The following sections describe the most relevant stages of our project.
In particular, \cref{chap:partI_intro} presents a panoramic of fluorescence microscopy, its application to natural sciences and the more general task of counting objects in images, as well as our contributions to such domains.
\Cref{chap:partI_dataset} then introduces the \textbf{Fluorescent Neuronal Cells} dataset with its characteristics and challenges.
\Cref{chap:partI_methods} discusses the alternative techniques we adopted to tackle the problem and the specifics of our experimental settings, with a particular focus on the \textbf{cell ResUnet} architecture of our proposal and the weight map adopted for promoting precise segmentation.
In \cref{chap:partI_results}, the output of our experiments is reported. Particular attention is devoted to comparing alternative approaches and evaluating various study design choices.
Finally, \cref{chap:partI_conclusions} summarizes the main findings of our work and outlines some possible future lines of research related to this application.