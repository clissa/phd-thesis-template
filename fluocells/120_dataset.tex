\chapter{Fluorescent neuronal cells dataset}
\label{chap:partI_dataset}

% \savegeometry{origigeom}
% \clearpage
% \newgeometry{lmargin=2cm}
\begin{figure}%[!b]
\begin{subfigure}{0.5\textwidth}
\includegraphics[width=\linewidth]{figures/120_dataset/i_empty.png}
\subcaption{empty}\label{fig:dataset:empty}
\end{subfigure}
\begin{subfigure}{0.5\textwidth}
\includegraphics[width=\linewidth]{figures/120_dataset/m_empty.png}
\subcaption{mask}\label{fig:dataset:empty_mask}
% \label{fig:dataset:empty}
\end{subfigure}

\begin{subfigure}{0.5\textwidth}
\includegraphics[width=\linewidth]{figures/120_dataset/i_168.jpeg}
\subcaption{dark}\label{fig:dataset:dark}
\end{subfigure}
\begin{subfigure}{0.5\textwidth}
\includegraphics[width=\linewidth]{figures/120_dataset/m_168.png}
\subcaption{mask}\label{fig:dataset:dark_mask}
% \label{fig:dataset:dark}
\end{subfigure}

\begin{subfigure}{0.5\textwidth}
\includegraphics[width=\linewidth]{figures/120_dataset/i_257.jpeg}
\subcaption{bright}\label{fig:dataset:bright}
\end{subfigure}
\begin{subfigure}{0.5\textwidth}
\includegraphics[width=\linewidth]{figures/120_dataset/m_257.png}
\subcaption{mask}\label{fig:dataset:bright_mask}
% \label{fig:dataset:bright}
\end{subfigure}
% \vspace{-0.2cm}
\caption{
\textbf{Sample data}. 
Raw images and corresponding ground-truth masks.
% The original images (left) present neuronal cells of different shape, size and saturation over a background of variable brightness and color.
% The corresponding ground-truth masks used for training (right) depicts cells as white pixels over a black background.
} \label{fig:dataset}
\end{figure}%
% \begin{figure}%[ht]\ContinuedFloat
% \centering
% \begin{subfigure}{0.5\textwidth}
% \includegraphics[width=\linewidth]{figures/120_dataset/i_clumping_yellow.png}
% \subcaption{}
% \end{subfigure}%
% \begin{subfigure}{0.5\textwidth}
% \includegraphics[width=\linewidth]{figures/120_dataset/m_clumping_yellow.png}
% \subcaption{}
% \label{fig:artifacts:clumping}
% \end{subfigure}

% \begin{subfigure}{0.5\textwidth}
% \includegraphics[width=\linewidth]{figures/120_dataset/i_252.jpeg}
% \subcaption{}
% \end{subfigure}%
% \begin{subfigure}{0.5\textwidth}
% \includegraphics[width=\linewidth]{figures/120_dataset/m_252.jpeg}
% \subcaption{}
% \label{fig:artifacts:stripe}
% \end{subfigure}%

% \begin{subfigure}{0.5\textwidth}
% \includegraphics[width=\linewidth]{figures/120_dataset/i_maccherone.jpeg}
% \subcaption{}
% \end{subfigure}%
% \begin{subfigure}{0.5\textwidth}
% \includegraphics[width=\linewidth]{figures/120_dataset/m_maccherone.png}
% \subcaption{}
% \label{fig:artifacts:macaroon}
% \end{subfigure}
% \vspace{-0.2cm}
% \caption{
% \textbf{Artifacts and challenges}. Neuronal cells appear of different shape, size and saturation over a background of variable brightness and color.
% %%rephrase
% % \textbf{Sample data}. In the original images (left), the neuronal cells of interest appear as yellow spots over a background of variable brightness and color. They exhibit a large variability in terms of shape, size and saturation, which makes them hard to distinguish from artifacts and similar biological structures that are not of interest.
% % The corresponding ground-truth masks used for training (right) depicts cells as white pixels over a black background.
% } 
% \label{fig:artifacts}
% \end{figure}
% \clearpage
% \restoregeometry

% \savegeometry{origigeom}
% \clearpage
% \newgeometry{lmargin=0.5cm}
% \begin{figure}%[ht]\ContinuedFloat
% \centering
% \begin{subfigure}{0.55\textwidth}
% % \subfloat[]{
% \includegraphics[width=\linewidth]{figures/120_dataset/annotated_i_clumping.png}
% \label{fig:artifacts:clumping}
% % }
% \subcaption{}
% \end{subfigure}%
% \begin{subfigure}{0.55\textwidth}
% % \subfloat[]{
% \includegraphics[width=\linewidth]{figures/120_dataset/annotated_i_clumping.png}
% \label{fig:artifacts:clumping}
% % }
% \subcaption{}
% \end{subfigure}
% \centering
% \begin{subfigure}{0.55\textwidth}
% % \subfloat[]{
% \includegraphics[width=\linewidth]{figures/120_dataset/annotated_i_stripe.png}
% \label{fig:artifacts:stripe}
% % }
% \subcaption{}
% \end{subfigure}%
% \begin{subfigure}{0.55\textwidth}
% % \subfloat[]{
% \includegraphics[width=\linewidth]{figures/120_dataset/annotated_i_macaroon.png}
% \label{fig:artifacts:macaroon}
% % }
% \subcaption{}
% \end{subfigure}

% \vspace{-0.2cm}
% \caption{
% \textbf{Artifacts and challenges}. Neuronal cells appear of different shape, size and saturation over a background of variable brightness and color.
% %%rephrase
% % \textbf{Sample data}. In the original images (left), the neuronal cells of interest appear as yellow spots over a background of variable brightness and color. They exhibit a large variability in terms of shape, size and saturation, which makes them hard to distinguish from artifacts and similar biological structures that are not of interest.
% % The corresponding ground-truth masks used for training (right) depicts cells as white pixels over a black background.
% } 
% \label{fig:artifacts2}
% \end{figure}
% \clearpage
% \restoregeometry

The \textbf{Fluorescent Neuronal Cells} dataset \cite{clissa2021fluocells} consists of 283 
% high-resolution 
pictures 
% (1600$\times$1200 pixels) 
of mice brain slices and the corresponding ground-truth labels.
In order to acquire these images, the mice were subjected to controlled experimental conditions, and a monosynaptic retrograde tracer (Cholera Toxin b, CTb) was surgically injected into brain tissues to highlight the neurons connected to the injection site
%projecting to the injection site
\cite{hitrec2019neural}.
Specimens of brain slices were then observed through  
a fluorescence microscope configured to select the narrow wavelength of light emitted by a fluorophore (of a yellow/orange color) associated with the tracer.
Hence, the resultant images depict neurons of interest as
% objects of different size and shape appearing as 
yellow-ish spots
% of variable brightness and saturation 
over a composite, generally darker background
% (\cref{fig:dataset,fig:artifacts}).
(\cref{fig:dataset:empty,fig:dataset:dark,fig:dataset:bright}).

% Although many efforts were made to stabilize the acquisition procedure, the images present several relevant challenges for the detection task. 
% For example, the variability in brightness and contrast causes some fickleness in the pictures overall appearance (cf. \cref{fig:dataset:dark,fig:dataset:bright}).  
% Also, the cells themselves exhibit varying saturation levels due to the natural fluctuation of the fluorescent emission properties (cf. \cref{fig:dataset:dark,fig:artifacts:clumping}).
% Moreover, the substructures of interest have a fluid nature. This implies that the size and shape of the stained cells may change significantly (see \cref{fig:artifacts:clumping}, right), making it even harder to discriminate between them and the background. 
% Combined to that, artifacts (\cref{fig:artifacts:stripe,fig:artifacts:macaroon}), bright biological structures -- like neurons' filaments -- (\cref{fig:artifacts:stripe}) and non-marked cells similar to the stained ones handicap the recognition task. 
% Last but not least, another source of complexity is the broad shift in the number of target cells from image to image.
% Indeed, the total counts range from no stained cells (\cref{fig:dataset:empty}) to several dozens clumping together (\cref{fig:artifacts:clumping}). 
% As a consequence, this requires a model with both high precision -- to prevent false positives in the former case -- and high recall -- since considering two or more touching neurons only once produces false negatives.
% % In the former case, the model needs high precision in order to prevent false positives. The latter, instead,
% % requires high recall since considering two or more touching neurons only once produces false negatives. 

% By and large, all of these factors make the recognition and counting tasks more problematic and complicate the model training.
% Likewise, these challenges hinder model evaluation as the interpretation of such borderline cases becomes subjective.

\input{fluocells/122_ground-truth_labels}

\section{Data exploration}

Fluorescent Neuronal Cells images are high-resolution RGB pictures of constant shape (1200 pixels height by 1600 pixels width) collected under fixed experimental conditions.
In terms of features, the data can be explored at two complementary levels. 
In fact, interesting insights can be retrieved by looking at the color and luminance information carried by pixels. Also, conducting analogous analyses on the ground-truth masks reveals

% In terms of data features, the most interesting aspects regard the color and luminance information, the counts distribution and the cells characteristics.
% In terms of data features, the most interesting aspects pertain the color and luminance information, the counts' distribution and characteristics of the cells.
The data can be explored at two complementary levels: pixels features and cells characteristics. 
On the one hand, interesting insights can be retrieved by looking at pixels' color and luminance information. Also, analogous analyses on the ground-truth masks reveal essential information about class-imbalance between signal and background.
On the other hand, examining object properties can highlight potential nuisances and suggest how to evaluate model performances.

\subsection{Graphical features}
% As far as the color, t
The picture appearance is dominated by two prevalent tints due to the intentional selection of a specific wavelength: a darker hue corresponding to areas whose light was filtered out and a yellow tone emitted by the fluorophore
(\cref{fig:dataset:empty,fig:artifacts:clumping}).
As a consequence, the only color channels to be populated are red and green, while blue is typically empty. 
An example of this effect is reported in \cref{fig:dataset:pixel_intensity}, where the average distribution of pixel intensity is illustrated.
\begin{figure}
    \centering
    \includegraphics[width=\textwidth]{figures/120_dataset/pixel_intensity_distribution.png}
    \caption{Pixel intensity distribution. Violin plot of the average distribution of pixel intensities across the  RGB channels}
    \label{fig:dataset:pixel_intensity}
\end{figure}
Guided by this observation, one may argue that all this information is superfluous and resorting to a grayscale representation could be better since the images are ultimately shades of yellow.
\Cref{fig:dataset:colorspace} (left) corroborates this intuition, as most pixels lay almost on a straight line in the red-green plane. 
This suggests that the two channels are highly correlated, so a one-dimensional subspace may be enough to represent most of the variability of the data.
In turn, this would bring two advantages: ease the learning process -- as neural networks typically suffer when inputs are correlated %\cite{} 
-- and make it more efficient -- as one considers only one channel instead of three.

However, the use-case at hand has no stringent requirements in terms of computing resources and runtime, so the 3-channels training is still feasible.
More importantly, the information thrown away when converting to grayscale, although tiny, may still be crucial to discriminate background and signal. 
At the same, the RGB colorspace may not be the optimal representation to learn this separation. For instance, \cref{fig:dataset:colorspace} (right) shows the same images according to a different encoding, HSV. 
In this case, the separation between dark and colored tones appears more evident. 
Moreover, most of the pixels are concentrated in low hue values
and their distribution seems more spread across the saturation-value plane. 

All that being considered, this work tried to leverage the insights of both approaches. 
On one side, the RGB colorspace was taken as a starting point to exploit all available information. On the other side, the model first layer was designed to incorporate a colorspace transformation from RGB to a single channel.
% In this way, the intent is to avoid introducing any colorspace-related bias by letting the model learn the most convenient representation and, at the same time, benefit from the computational advantage due to a lower dimensionality.
The intent is to avoid introducing any colorspace-related bias by letting the model learn the most convenient representation without ignoring the fact that a one-dimensional manifold is probably enough to express the variability of the data.
\begin{figure}
    \centering
    \includegraphics[width=1.1\textwidth]{figures/120_dataset/colorspace_Mar23bS1C2R3_VLPAGl_200x_y.png}
    
    \centering
    \includegraphics[width=1.1\textwidth]{figures/120_dataset/colorspace_Mar26bS2C1R2_DMl_200x_y.png}
    \caption{Colorspace representation. The same image is represented as RGB (left) and HSV (right). Pixels are treated as 3D points with coordinates given by their encoding in the corresponding colorspace}
    \label{fig:dataset:colorspace}
\end{figure}

\subsection{Class imbalance}


\subsection{Objects features}



\section{Challenges}

% Although many efforts were made to stabilize the acquisition procedure, the images present several relevant challenges for the detection task. 
% For example, the variability in brightness and contrast causes some fickleness in the pictures overall appearance (cfr. \cref{fig:dataset:dark,fig:dataset:bright}).  
% Also, the cells themselves exhibit varying saturation levels due to the natural fluctuation of the fluorescent emission properties (cfr. \cref{fig:dataset:dark,fig:artifacts:clumping}).
% Moreover, the substructures of interest have a fluid nature. This implies that the size and shape of the stained cells may change significantly (see \cref{fig:artifacts:clumping}, right), making it even harder to discriminate between them and the background. 
% Combined to that, artifacts (\cref{fig:artifacts:stripe,fig:artifacts:macaroon}), bright biological structures -- like neurons' filaments -- (\cref{fig:artifacts:stripe}) and non-marked cells similar to the stained ones handicap the recognition task. 
% Last but not least, another source of complexity is the broad shift in the number of target cells from image to image.
% Indeed, the total counts range from no stained cells (\cref{fig:dataset:empty}) to several dozens clumping together (\cref{fig:artifacts:clumping}). 
% As a consequence, this requires a model with both high precision -- to prevent false positives in the former case -- and high recall -- since considering two or more touching neurons only once produces false negatives.
% % In the former case, the model needs high precision in order to prevent false positives. The latter, instead,
% % requires high recall since considering two or more touching neurons only once produces false negatives. 

% \note[Luca][notesyellow]{Aggiungere unbalanced dataset}

Although many efforts were made to stabilize the acquisition procedure, the images present several relevant challenges for the detection task. 

Graphical properties of the pictures are undoubtedly valuable information for the classification of the pixels. However, that alone is not enough.
Indeed, an elementary approach would be to apply selection cuts over these features to separate cell pixels from the background.
Nonetheless, the characterization of cells in terms of color, saturation and contrast varies from image to image, making it difficult to generalize hard-coded thresholds.
For example, the tissues can sometimes soak in some of the marker, causing irrelevant compounds to emit light which is then captured by the microscope. 
When that is the case, the background assumes a similar hue to the neuronal cells (cfr. \cref{fig:dataset:dark,fig:dataset:bright}), and the identification falls back to other characteristics such as saturation and contrast.
In addition to that, fluorescent emissions are naturally unstable, thus generating fluctuations of the saturation levels exhibited by pixels belonging to cells (cfr. \cref{fig:dataset:dark,fig:artifacts:clumping}).

Moreover, the substructures of interest have a fluid nature. This implies that the size and shape of the stained cells may change significantly (see \cref{fig:artifacts:clumping}, right and \cref{fig:dataset:geom_area}), making it even harder to discriminate between them and the background.

Another challenge is represented by artifacts, i.e. fictitious objects accidentally present in pictures which resemble neuronal cells in terms of shape, size or color.
One example is when the marker forms accumulations of fluorophore in narrow areas that generate emissions very similar to the ones of the cells. In such cases, thresholding becomes ineffective and one has to resort to physiological characteristics as cells' shape and size in order to distinguish them from such artifacts.
Nevertheless, the distinction is not always unambiguous and this poses an issue of intrinsic subjectivity in the annotation process, which is then reflected on model performance.

A further source of complexity is represented by the broad shift in the number of target cells from image to image.
Indeed, the total counts range from no stained cells (\cref{fig:dataset:empty}) to several dozens clumping together (\cref{fig:artifacts:clumping}). 
As a consequence, this requires a model with both high precision -- to prevent false positives in the former case -- and high recall -- since considering two or more touching neurons only once produces false negatives.

Last but not least, the objects are typically small and cover only marginal portions of the images. This generates an extreme imbalance between signal and background, which is even worsened by the high resolution of the pictures.
Hence, dedicated learning strategies are demanded to mitigate this issue during the training phase.

By and large, all of these factors make the recognition and counting tasks more problematic and complicate the learning process.
Likewise, borderline annotations hinder model evaluation as their subjectivity deprives the model of a reliable and indisputable testbed.
% \begin{figure}%[!b]
% \begin{subfigure}{0.5\textwidth}
% \includegraphics[width=\linewidth]{figures/120_dataset/i_empty.png}
% \subcaption{}
% \end{subfigure}%
% \begin{subfigure}{0.5\textwidth}
% \includegraphics[width=\linewidth]{figures/120_dataset/m_empty.png}
% \subcaption{}
% \label{fig:dataset:empty}
% \end{subfigure}

% \centering
% \begin{subfigure}{0.5\textwidth}
% \includegraphics[width=\linewidth]{figures/120_dataset/i_168.jpeg}
% \subcaption{}
% \end{subfigure}%
% \begin{subfigure}{0.5\textwidth}
% \includegraphics[width=\linewidth]{figures/120_dataset/m_168.png}
% \subcaption{}
% \label{fig:dataset:dark}
% \end{subfigure}

% \centering
% \begin{subfigure}{0.5\textwidth}
% \includegraphics[width=\linewidth]{figures/120_dataset/i_257.jpeg}
% \subcaption{}
% \end{subfigure}%
% \begin{subfigure}{0.5\textwidth}
% \includegraphics[width=\linewidth]{figures/120_dataset/m_257.png}
% \subcaption{}
% \label{fig:dataset:bright}
% \end{subfigure}
% \vspace{-0.2cm}
% \caption{
% \textbf{Sample data}. 
% Original images (left) and corresponding ground-truth masks (right).
% % The original images (left) present neuronal cells of different shape, size and saturation over a background of variable brightness and color.
% % The corresponding ground-truth masks used for training (right) depicts cells as white pixels over a black background.
% } \label{fig:dataset}
% \end{figure}%
% \begin{figure}%[ht]\ContinuedFloat
% \centering
% \begin{subfigure}{0.5\textwidth}
% \includegraphics[width=\linewidth]{figures/120_dataset/i_clumping_yellow.png}
% \subcaption{}
% \end{subfigure}%
% \begin{subfigure}{0.5\textwidth}
% \includegraphics[width=\linewidth]{figures/120_dataset/m_clumping_yellow.png}
% \subcaption{}
% \label{fig:artifacts:clumping}
% \end{subfigure}

\begin{landscape}
\begin{figure}
    \centering
    \includegraphics[width=\linewidth]{figures/120_dataset/challenges/challenges.pdf}
    \caption{Caption}
    \label{fig:artifacts:clumping}
\end{figure}
\end{landscape}

\begin{landscape}
\begin{figure}
    \centering
    \includegraphics[width=\linewidth]{figures/120_dataset/challenges/stripe_and_filaments.pdf}
    \caption{Caption}
    \label{fig:artifacts:stripe}
\end{figure}
\end{landscape}

\begin{landscape}
\begin{figure}
    \centering
    \includegraphics[width=\linewidth]{figures/120_dataset/challenges/artifacts.pdf}
    \caption{Caption}
    \label{fig:artifacts:macaroon}
\end{figure}
\end{landscape}
