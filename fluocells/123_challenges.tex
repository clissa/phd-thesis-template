\section{Challenges}

Although many efforts were made to stabilize the acquisition procedure, the images present several relevant challenges for the detection task. 
For example, the variability in brightness and contrast causes some fickleness in the pictures overall appearance (cfr. \cref{fig:dataset:dark,fig:dataset:bright}).  
Also, the cells themselves exhibit varying saturation levels due to the natural fluctuation of the fluorescent emission properties (cfr. \cref{fig:dataset:dark,fig:artifacts:clumping}).
Moreover, the substructures of interest have a fluid nature. This implies that the size and shape of the stained cells may change significantly (see \cref{fig:artifacts:clumping}, right), making it even harder to discriminate between them and the background. 
Combined to that, artifacts (\cref{fig:artifacts:stripe,fig:artifacts:macaroon}), bright biological structures -- like neurons' filaments -- (\cref{fig:artifacts:stripe}) and non-marked cells similar to the stained ones handicap the recognition task. 
Last but not least, another source of complexity is the broad shift in the number of target cells from image to image.
Indeed, the total counts range from no stained cells (\cref{fig:dataset:empty}) to several dozens clumping together (\cref{fig:artifacts:clumping}). 
As a consequence, this requires a model with both high precision -- to prevent false positives in the former case -- and high recall -- since considering two or more touching neurons only once produces false negatives.
% In the former case, the model needs high precision in order to prevent false positives. The latter, instead,
% requires high recall since considering two or more touching neurons only once produces false negatives. 

\note[Luca][notesyellow]{Aggiungere unbalanced dataset}

By and large, all of these factors make the recognition and counting tasks more problematic and complicate the model training.
Likewise, these challenges hinder model evaluation as the interpretation of such borderline cases becomes subjective.