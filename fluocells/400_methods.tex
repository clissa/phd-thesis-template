\chapter{Methods}
\label{chap:partI_methods}

This chapter discusses a deep learning attempt to tackle the problem of segmenting and counting cells in microscopic fluorescence using a \textit{supervised learning} framework. 
For this purpose, we leverage the recently proposed \textbf{cell-ResUnet} architecture \cite{morelli2021cresunet} for object segmentaion, and we adopt careful training strategies to address the principal challenges of the Fluorescent Neuronal  Cells dataset \cite{clissa2021fluocells} described in \cref{sec:challenges}.
Once the cells are detected, the final count is retrieved as the number of connected pixels in the post-processed output.
In order to demonstrate the efficacy of our method, we compare the cell-ResUnet against three CNN architectures belonging to the Unet and ResUnet families. 
In doing so, we also test the impact of study design choices intended to reduce false negatives and promote accurate segmentation.
In addition to that, an adaptive thresholding approach is also tested as a baseline for performance measurement. 





