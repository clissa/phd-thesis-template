\section{Non-ML baseline}
\label{baseline}

Machine and deep learning have succeeded in many applications from several domains lately, thus building great expectations and becoming hot topics in current innovation processes at various societal levels.
Nevertheless, these powerful techniques are not a magic bullet to solve any data-related problem \cite{wolpert1997nofreelunch}. They come with their own challenges and limitations that are often overlooked in real-world applications, possibly causing thunderous failures due to unjustified expectations.
In fact, it is not unusual to fall victim to their popularity only to find months down the line that more straightforward methods work best for some specific task or data.

To avoid incurring this ``ML hype curse", this work considers a simple \mbox{non-ML} approach as a baseline. 
In particular, this consists in an adaptive thresholding mechanism implemented to exploit the pixel intensity information for binarization.
In practice, the input image is read as grayscale and a selection cutoff is set at a configurable quantile of the pixel intensity distribution.
A binary mask is then obtained by labeling pixels above that threshold as cells and the rest as background.
After that, the same post-processing steps as the ones adopted for the output of the ML models are applied (see \cref{sec:post_processing}).
This operation is performed for all training and validation pictures, and the goodness of fit is assessed as described in \cref{sec:model_evaluation}.
The whole procedure is then repeated varying the quantile value used for thresholding. A starting search is conducted using a coarse grid from 0.9 to 0.99 with steps of 0.01. This is intended to explore the hyperparameter space and get an idea of where the approach performs best. 
Since this happens for higher values, a finer grid from 0.97 to 0.999 with steps of 0.001 is exploited for fine-tuning.
Finally, the cutoff corresponding to the highest $F_1$ score is chosen as the optimal threshold and is later used to assess the baseline performance on the test set.