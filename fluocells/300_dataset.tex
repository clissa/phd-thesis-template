\chapter{Fluorescent neuronal cells dataset}
\label{chap:partI_dataset}

% \savegeometry{origigeom}
% \clearpage
% \newgeometry{lmargin=2cm}
\begin{figure}%[!b]
\begin{subfigure}{0.5\textwidth}
\includegraphics[width=\linewidth]{figures/120_dataset/i_empty.png}
\subcaption{empty}\label{fig:dataset:empty}
\end{subfigure}
\begin{subfigure}{0.5\textwidth}
\includegraphics[width=\linewidth]{figures/120_dataset/m_empty.png}
\subcaption{mask}\label{fig:dataset:empty_mask}
% \label{fig:dataset:empty}
\end{subfigure}

\begin{subfigure}{0.5\textwidth}
\includegraphics[width=\linewidth]{figures/120_dataset/i_168.jpeg}
\subcaption{dark}\label{fig:dataset:dark}
\end{subfigure}
\begin{subfigure}{0.5\textwidth}
\includegraphics[width=\linewidth]{figures/120_dataset/m_168.png}
\subcaption{mask}\label{fig:dataset:dark_mask}
% \label{fig:dataset:dark}
\end{subfigure}

\begin{subfigure}{0.5\textwidth}
\includegraphics[width=\linewidth]{figures/120_dataset/i_257.jpeg}
\subcaption{bright}\label{fig:dataset:bright}
\end{subfigure}
\begin{subfigure}{0.5\textwidth}
\includegraphics[width=\linewidth]{figures/120_dataset/m_257.png}
\subcaption{mask}\label{fig:dataset:bright_mask}
% \label{fig:dataset:bright}
\end{subfigure}
% \vspace{-0.2cm}
\caption{
\textbf{Sample data.} 
Raw images and corresponding ground-truth masks.
% The original images (left) present neuronal cells of different shape, size and saturation over a background of variable brightness and color.
% The corresponding ground-truth masks used for training (right) depicts cells as white pixels over a black background.
} \label{fig:dataset}
\end{figure}%
% \begin{figure}%[ht]\ContinuedFloat
% \centering
% \begin{subfigure}{0.5\textwidth}
% \includegraphics[width=\linewidth]{figures/120_dataset/i_clumping_yellow.png}
% \subcaption{}
% \end{subfigure}%
% \begin{subfigure}{0.5\textwidth}
% \includegraphics[width=\linewidth]{figures/120_dataset/m_clumping_yellow.png}
% \subcaption{}
% \label{fig:artifacts:clumping}
% \end{subfigure}

% \begin{subfigure}{0.5\textwidth}
% \includegraphics[width=\linewidth]{figures/120_dataset/i_252.jpeg}
% \subcaption{}
% \end{subfigure}%
% \begin{subfigure}{0.5\textwidth}
% \includegraphics[width=\linewidth]{figures/120_dataset/m_252.jpeg}
% \subcaption{}
% \label{fig:artifacts:stripe}
% \end{subfigure}%

% \begin{subfigure}{0.5\textwidth}
% \includegraphics[width=\linewidth]{figures/120_dataset/i_maccherone.jpeg}
% \subcaption{}
% \end{subfigure}%
% \begin{subfigure}{0.5\textwidth}
% \includegraphics[width=\linewidth]{figures/120_dataset/m_maccherone.png}
% \subcaption{}
% \label{fig:artifacts:macaroon}
% \end{subfigure}
% \vspace{-0.2cm}
% \caption{
% \textbf{Artifacts and challenges}. Neuronal cells appear of different shape, size and saturation over a background of variable brightness and color.
% %%rephrase
% % \textbf{Sample data}. In the original images (left), the neuronal cells of interest appear as yellow spots over a background of variable brightness and color. They exhibit a large variability in terms of shape, size and saturation, which makes them hard to distinguish from artifacts and similar biological structures that are not of interest.
% % The corresponding ground-truth masks used for training (right) depicts cells as white pixels over a black background.
% } 
% \label{fig:artifacts}
% \end{figure}
% \clearpage
% \restoregeometry

% \savegeometry{origigeom}
% \clearpage
% \newgeometry{lmargin=0.5cm}
% \begin{figure}%[ht]\ContinuedFloat
% \centering
% \begin{subfigure}{0.55\textwidth}
% % \subfloat[]{
% \includegraphics[width=\linewidth]{figures/120_dataset/annotated_i_clumping.png}
% \label{fig:artifacts:clumping}
% % }
% \subcaption{}
% \end{subfigure}%
% \begin{subfigure}{0.55\textwidth}
% % \subfloat[]{
% \includegraphics[width=\linewidth]{figures/120_dataset/annotated_i_clumping.png}
% \label{fig:artifacts:clumping}
% % }
% \subcaption{}
% \end{subfigure}
% \centering
% \begin{subfigure}{0.55\textwidth}
% % \subfloat[]{
% \includegraphics[width=\linewidth]{figures/120_dataset/annotated_i_stripe.png}
% \label{fig:artifacts:stripe}
% % }
% \subcaption{}
% \end{subfigure}%
% \begin{subfigure}{0.55\textwidth}
% % \subfloat[]{
% \includegraphics[width=\linewidth]{figures/120_dataset/annotated_i_macaroon.png}
% \label{fig:artifacts:macaroon}
% % }
% \subcaption{}
% \end{subfigure}

% \vspace{-0.2cm}
% \caption{
% \textbf{Artifacts and challenges}. Neuronal cells appear of different shape, size and saturation over a background of variable brightness and color.
% %%rephrase
% % \textbf{Sample data}. In the original images (left), the neuronal cells of interest appear as yellow spots over a background of variable brightness and color. They exhibit a large variability in terms of shape, size and saturation, which makes them hard to distinguish from artifacts and similar biological structures that are not of interest.
% % The corresponding ground-truth masks used for training (right) depicts cells as white pixels over a black background.
% } 
% \label{fig:artifacts2}
% \end{figure}
% \clearpage
% \restoregeometry

The \textbf{Fluorescent Neuronal Cells} dataset \cite{clissa2021fluocells} consists of 283 
% high-resolution 
pictures 
% (1600$\times$1200 pixels) 
of mice brain slices and the corresponding ground-truth labels.
In order to acquire these images, the mice are first subjected to controlled experimental conditions. Then, a monosynaptic retrograde tracer (b-subunit of Cholera Toxin, CTb) is surgically injected into brain tissues of interest -- i.e. dorsomedial hypothalamic nucleus (DM), lateral hypothalamic area (LH) and ventrolateral part of the periaqueductal gray matter (VLPAG) -- to mark the neurons projecting into the injection site \cite{hitrec2019neural}, thus highlighting functional connections between brain regions.
These tracers are widely used to understand neuronal links among neural structures since their composition allows synaptic termini to capture the tracer and then retrogradely transport it into the cell soma through the axon. As the term ``monosynaptic'' suggests, they cannot pass into other cells once captured by the synaptic terminal, thus enabling a unique association between the marked neurons and the area where the tracer was injected. 
After some time required for capturing and transporting the tracer -- 7 days in our case, as typical for laboratory rodents --, the brain is cut into slices and observed through a fluorescence microscope.
This tool allows to lit the specimens with light at a specific wavelength and select the corresponding narrow frequency emitted by a fluorophore %(of yellow/orange color)
associated with the tracer. 
In our case, the samples are observed at a 200x zoom through a Nikon eclipse 80i microscope equipped with a Nikon Digital Sight DS-Vi1 color camera (3.1876 pixels/micron of resolution).
The configuration adopted exploits absorption (excitation) and emission wavelengths of 555 and 580 nanometers, respectively, corresponding to a hue between yellow and orange.
Thus, the resulting images depict neurons of interest as
% objects of different size and shape appearing as 
yellow-ish spots
% of variable brightness and saturation 
over a composite, generally darker background
% (\cref{fig:dataset,fig:artifacts}).
(\cref{fig:dataset:empty,fig:dataset:dark,fig:dataset:bright}).

% Although many efforts were made to stabilize the acquisition procedure, the images present several relevant challenges for the detection task. 
% For example, the variability in brightness and contrast causes some fickleness in the pictures overall appearance (cf. \cref{fig:dataset:dark,fig:dataset:bright}).  
% Also, the cells themselves exhibit varying saturation levels due to the natural fluctuation of the fluorescent emission properties (cf. \cref{fig:dataset:dark,fig:artifacts:clumping}).
% Moreover, the substructures of interest have a fluid nature. This implies that the size and shape of the stained cells may change significantly (see \cref{fig:artifacts:clumping}, right), making it even harder to discriminate between them and the background. 
% Combined to that, artifacts (\cref{fig:artifacts:stripe,fig:artifacts:macaroon}), bright biological structures -- like neurons' filaments -- (\cref{fig:artifacts:stripe}) and non-marked cells similar to the stained ones handicap the recognition task. 
% Last but not least, another source of complexity is the broad shift in the number of target cells from image to image.
% Indeed, the total counts range from no stained cells (\cref{fig:dataset:empty}) to several dozens clumping together (\cref{fig:artifacts:clumping}). 
% As a consequence, this requires a model with both high precision -- to prevent false positives in the former case -- and high recall -- since considering two or more touching neurons only once produces false negatives.
% % In the former case, the model needs high precision in order to prevent false positives. The latter, instead,
% % requires high recall since considering two or more touching neurons only once produces false negatives. 

% By and large, all of these factors make the recognition and counting tasks more problematic and complicate the model training.
% Likewise, these challenges hinder model evaluation as the interpretation of such borderline cases becomes subjective.
