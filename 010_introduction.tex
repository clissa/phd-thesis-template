\chapter{Introduction}
\label{chap:introduction}


\emph{Data Science} is a very vibrant field of research that has been gaining more and more interest in the past decade, both in academy and industry. 
\begin{figure}
\centerline{
\includegraphics[width=0.5\textwidth]{figures/DataScience/trend_ds.png}
\includegraphics[width=0.5\textwidth]{figures/DataScience/map_ds.png}
}
\caption{
\textbf{Data Science Google searches}. Trends and geolocalization of Google searches of "data science" from 01/01/04 to 06/12/21.
} 
\label{fig:GoogleTrendsDS}
\end{figure}
That much so that it was awarded the title of \textquote{sexiest job of the 21st century} \cite{davenport2012sexiest}, and only American universities counted 78 data science programs in 2020 \cite{zhang2021data}.
However, the discussion over data science's essence has a long history, and multiple definitions have been proposed over the years \cite{donoho201750years}.
Although researchers and practitioners are yet to reach a complete agreement on its exact meaning  \cite{ASA2015statement}, \emph{five} common pillars can be identified by the various definitions.
First, \textbf{multidisciplinarity} is indisputably a key element stressed in every definition of data science. 
Second, as the name suggests, the \textbf{focus on data} and adequate techniques to manage and process them is inevitably an essential aspect.
Third, data science requires adopting suitable \textbf{analytical models} to transform data into knowledge.
Fourth, the \textbf{computing infrastructure} that is necessary to run data analysis timely and efficiently. 
Fifth, a compelling \textbf{visualization and communication} of the results that are simple enough to speak to a heterogeneous and non-technical audience, yet comprehensive of all relevant details to convey meaningful insights.

Inspired by these principles, this thesis describes the development of two data science projects and how the five pillars above are declined in practice.


\paragraph{Structure of the Thesis}

After an initial definition of the discipline of \emph{Data Science}, this thesis is organized as follows.

\Cref{chap:historyDS} draws a historical reconstruction of the evolution of the concept of data science over time, trying to clarify what this subject is all about and set an unambiguous reference framework. 
\note{TO BE COMPLETED WITH WHOLE STRUCTURE.}

\input{011_HistoryOfDataScience}
