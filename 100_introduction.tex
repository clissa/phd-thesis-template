\chapter{Introduction}
\label{chap:introduction}


\textit{Data Science} is a very vibrant field of research that has been gaining more and more interest across all the world in the past decade (\cref{fig:GoogleTrendsDS}), both in the educational and industrial sectors.
\begin{figure}
\centering
\subfloat[time series]{
\includegraphics[width=\textwidth]{figures/DataScience/googleTrendsTS.pdf}\label{fig:GoogleTrendsDS:ts}
}

\subfloat[geolocalization]{
\includegraphics[width=\textwidth]{figures/DataScience/googleTrendsMap.pdf}\label{fig:GoogleTrendsDS:map}
}
\caption{
\textbf{``Data Science" Google searches}. The trend of Google searches for the term ``data science" (\hyperref[fig:GoogleTrendsDS:ts]{a}) and the corresponding geolocalization (\hyperref[fig:GoogleTrendsDS:map]{b}) show an evident increase of the popularity of the subject and its global penetration
} 
\label{fig:GoogleTrendsDS}
\end{figure}
That much so that it was awarded the title of \textquote{sexiest job of the 21st century} \cite{davenport2012sexiest}, and only American universities counted 78 data science programs in 2020 \cite{zhang2021data}.

However, the discussion over data science's essence has a long history, and multiple definitions have been proposed over the years \cite{donoho201750years}.
Although researchers and practitioners are yet to reach a complete agreement on its exact meaning  \cite{ASA2015statement}, \emph{five} common pillars can be identified by the various definitions.
First, \textbf{multidisciplinarity} is indisputably a key element stressed in every definition of data science. 
Second, as the name suggests, the \textbf{focus on data} and adequate techniques to manage and process them is inevitably an essential aspect.
Third, data science requires adopting suitable \textbf{statistical models} to convert data into knowledge.
Fourth, the \textbf{computing infrastructure} that is necessary to manage the data and analyze them efficiently. 
Fifth, a compelling \textbf{visualization and communication} of results that is simple enough to be understood by a heterogeneous and non-technical audience, yet comprehensive of all relevant details to convey valuable insights.

Inspired by these principles, we reckon that the essence of data science ought to be traced back to its practical aspect, which is the starting sparkle and the ultimate destination of any data science investigation.
Indeed, all data science projects share the common target of solving real-life problems in a data-driven fashion.
Although overcoming hampers is not a novel challenge -- in fact, one may see it as a key aspect of the evolution really, which has always been crucial for the progression of species, especially regarding the human kind -- the peculiar feature that distinguishes data science is its framing into a data-oriented paradigm.
Not surprisingly, its explosion comes as a normal response to the increased availability of data we have been experiencing in modern times.
Having data about virtually any aspect of life poses the question of how to leverage this information to tackle unsolved problems and re-think current approaches.
This in turn urges for a methodological framework that provides tools for modeling natural phenomena and learning from their data.
As a consequence, computing infrastructures are needed to manage the data and enable the practical implementation and use of the developed theories.
Hence, the foundational core of data science can be established in the trinomial \textit{\mbox{data-modeling-computing}} that summarizes the practical, theoretical and technical challenges related to the discipline.
% \sidenote[Luca][notesyellow]{alternative trinomio: 1) "data-modeling-computing", 2) "data-learning-computing"}
Importantly, the above chain of causes-consequences is not static, as advancements in any of the three areas pave the way for novel research in the others. 
Therefore, each of the components can be seen as both an enabling factor and a stimulus for the others, thus generating a virtuous circle that encourages progress.
% \sidenote[Luca][notesyellow]{Inserire schema circolo virtuoso}
% \sidenote[Luca][notesyellow]{procedere esempi di big data + ML e computing come un enabling factor per la ricerca}
% \note[Luca][notesyellow]{
% The modern popularity of neural networks is a vivid example of how this collaborative interchange acts in practice.
% Born as an altenative way to analyze data \protect \cite{mitchell1989}, they were long time under the radars.
% However, the sudden availability of more data and more powerful computing resources became an enabling factor for their adoption and practical successes.
% This in turn generated the curiosity to test that methodology on a cascade of what-if scenarios:
% bigger datasets, rephrasal of the tasks, ... .
% At the same time, this stimulated a correspondent urge for more efficient and sustainable computing approaches.
% From which it came new computing paradigms (distributed, hadoop, spark, gpu accelerators, ...)
% Which generated new approaches (CV, NLP, ....)

% Which generates the questions on how to collect better data (active learning or CL), and yet more efficient computing solutions (QC).
% This stimulates the question of how to leverage these new solutions, for instance people are studying how to translate classical machine learning building blocks exploiting the computational boost of quantum devices.
% }
Clearly, this process typically requires mixing heterogeneous competencies, from which the necessity for contamination between multiple disciplines contributing to sub-tasks propaedeutic to the final solution.
Also, a clear visualization and communication of the results is paramount to guarantee effective usage of the developed methodologies and avoid misunderstandings in the various inter-domain interactions.
% As a result, a clear visualization and communication is paramount to avoid misunderstandings in the various inter-domain interactions and guarantee effective usage of the developed approaches in practice.
% Guided by the former vision, we frame data science as an applied science that studies how to combine domain expertise with suitable learning approaches and necessary computing resources to process real data for a tangible return.

% Hence, the collection of data and the development of proper methodologies to collect and analyze them justifies the need for statistical theory.
% However, this requires enough computing power to store and process the collected data.
% In turn, the advancement in computer technologies has enabled new paradigms to manage huge amounts of data, and made statistical models feasible, thus also stimulating new research in these fields, that require more computing thus creating a vicious circle that encourage the three discipline to progress.
% Not surprisingly, its explosion comes as a normal response to the increased availability of data about virtually any aspect of life that has led to the so-call big data era.
% Besides that, the progress of computing technologies has been a key enabling factor to allow one one side to store and manage huge amounts of data.
% Also, the enhanced computing power has has made 
% stimulated research on statistical methods to analyze increased quantities of data that could not be possible without sufficient computing power.
% \sidenote[Luca][notesyellow]{Circolo vizioso: menzionare trinomio "data-statistics-computing" e se possibile schema}

% However, the above schema requires complex interaction between different areas, each required profound proficiency to be carried out effectively to produce a valuable result in a timely manner.
% Hence, the need for suitable visualizations that put the last tile to reach valuable insights as a result.
% One of the enabling factors 
% One of these components is surely data. The big data era we are witnessing entails a 
% Also, the increasing availabilty of data about virtually any aspect of life, nowadays allows to tackle problems in a more systematic and critic way, thus the data-driven strategy. 
% Hence, the other pillars come as fundamental 


Guided by the former vision, this work discusses two independent data science projects developed in the scope of multidisciplinary applied research, and describes how the five pillars above are declined for the two use cases.
Despite treating the proposed methodologies in detail, a particular emphasis posits on \textit{i)} the interdisciplinary nature of the projects, \textit{ii)} the origin of the data, the information they carry and the corresponding challenges, and \textit{iii)} the practical impact of our approaches.
Also, specific attention is devoted to data visualization, both in the initial exploration and the presentation of the results, for its crucial role in enabling the communication among experts from various domains.

\subsection*{Structure of the Thesis}

After an initial definition of the discipline of \emph{Data Science}, this thesis is organized as follows.

\Cref{sec:historyDS} draws a historical reconstruction of the evolution of the data science concept over time, trying to clarify what this subject is all about and set an unambiguous reference framework. 
\Cref{sec:learning} then introduces the major learning paradigms and their characteristics.

\Cref{partI} explores in greater detail the work presented in \citeA{morelli2021cresunet}. In particular, \Cref{chap:partI_intro} describes the technique of microscopic fluorescence and its application to life science and biology experiments. The task of counting objects in digital images is then presented, and some relevant literature is reviewed.
\Cref{chap:partI_dataset} describes the \textbf{Fluorescent Neuronal Cells} dataset \cite{clissa2021fluocells}, focusing on data acquisition, data annotations, and peculiar characteristics and challenges. 
In \Cref{chap:partI_methods}, the \textbf{cell ResUnet (c-ResUnet)} \cite{morelli2021cresunet} model is introduced and compared with several alternative architectures. Also, three experimental settings are detailed, which are then used for testing competing architectures through ablation studies.
In \Cref{chap:partI_results}, the performances achieved by the proposed approaches are evaluated both quantitatively and qualitatively.
Finally, \Cref{chap:partI_conclusions} summarizes the main findings of the study and discusses possible extensions.

\Cref{partII} deals with the issue of computing infrastructure management in the context of the Worldwide Large Hadron Collider Computing Grid.
\Cref{ch:opint_intro} introduces the High-Energy Physics community and the related experiments conducted through the Large Hadron Collider (LHC) at the CERN laboratory, with a particular focus on the computing infrastructure adopted for sharing and analyzing the experimental data. 
In particular, \cref{ch:opint} describes more in detail the data flows generated by the LHC and introduces the Operational Intelligence project \cite{opint2020}, a joint effort of different collaborations to automate infrastructure management.
Among its various activities, the subject of troubleshooting transfer failures is taken into account, and some ongoing attempts are outlined alongside our contributions.
\Cref{ch:opint_methods} discusses a possible approach to the problem aimed at supporting current operations by automatically extracting a few error categories (in the order of tens) from the massive amount of daily failed transfers (in the order of millions).
\Cref{ch:opint_results} reports a demonstration of our approach applied to one day of data as a proof of concept for its potential, taking into account both a qualitative interpretation of the results and a quantitative proxy of the performance.
Finally, \cref{ch:opint_conclusions} summarizes the advantages and limitations of the proposed methodology and suggests a possible direction for future developments.

\section{History of Data Science}
\label{sec:historyDS}

Some authors attribute the first appearance of the term ``data science" in literature
% (e.g. \citeNP{cao2017data})
to \citeA{naur1974concise}, where he presents data science as an alternative name for computer science and defines it as \textquote{\textit{the science of dealing with data, once they have been established, while the relation of the data to what they represent is delegated to other fields and sciences}}.


% The first trace of the debate about data science dates back to \citeA{tukey1962future}, where the term \emph{data analysis} is used to indicate a discipline with the connotations of a \emph{science} and which is \textquote{defined by a ubiquitous problem rather than by a concrete subject}. 
However, the first mention of something resembling what we currently perceive as data science dates back to \citeA{tukey1962future}, where the term \emph{data analysis} is used to indicate a discipline with the connotations of \emph{science}, which is \textquote{\textit{defined by a ubiquitous problem rather than by a concrete subject}}. 
Tukey's description incorporates many aspects seemingly tied closely to applied statistics: 
\begin{displayquote}
procedures for analyzing data, techniques for interpreting the results of such procedures, ways of planning the gathering of data to make its analysis easier, more precise or more accurate, and all the machinery and results of (mathematical) statistics which apply to analyzing data.
\end{displayquote}
Nevertheless, the extent Tukey attributes to data analysis is broader than its philological meaning, as it comprises all of statistics and embeds it in a larger entity \cite{huber2012data, donoho201750years}.
Indeed, Tukey himself sets the boundaries between data analysis and statistics in their respective binding to the strict formalism of mathematics.
In fact, he appeals for a looser attachment to mathematical rigor and suggests focusing on actionable insights rather than theory (\textquote{\textit{scope and usefulness over security}}).
In Tukey's vision, data analysis is framed within the paradigm of an empirical science proceeding by trial and error, whose role is to advise actions and insights for new investigations.
More precisely, this has to hold even when not enough evidence is available to be (mathematically) sure of the conclusions, thus accepting a reasonable degree of error (\textquote{\textit{Data analysis must be willing to err moderately often in order that inadequate evidence shall more often suggest the right answer}}).
As a result, the mathematical foundations should be used as an element to directing the strategy of investigation rather than validating its results (\textquote{\textit{as bases for judgment rather than as bases for proof}}).
Finally, Tukey identifies four driving forces in the emerging data analysis science:
\begin{displayquote}
Four major influences act on data analysis today:
\begin{enumerate}
\setstretch{0.3}
    \item The formal theories of statistics
    \item Accelerating developments in computers and display devices
    \item The challenge, in many fields, of more and ever larger bodies of data
    \item The emphasis on quantification in an ever wider variety of disciplines
\end{enumerate}
\end{displayquote}
Despite being released 60 years ago, Tukey's description is astonishingly modern and well depicts many activities under the umbrella of what we refer to as data science today.
In line with Tukey's vision, several authors have defined -- more or less explicitly -- data science as an extension of applied statistics.

In 1985, Wu used the term \textquote{Data Science} for the first time as an alternative name for statistics, and in 1997 advocated the renaming \textquote{\textit{Statistics $=$ Data Science}} to avoid the limiting association of statistics with accounting and data description.
In particular, he indicates \textit{large/complex data}, \textit{data-driven methodologies} and \textit{representation and exploitation of knowledge} as promising future directions for statistics. Also, he suggests that statistics degree programs should be grounded on the \textquote{\textit{Statistical Trilogy: Data Collection -- Modeling and Analysis -- Problem Understanding/Solving and Decision Making}}, with a strong emphasis on a balanced, interdisciplinary curriculum \cite{wu1997statistic}.

In 1993, Chambers provocatively distinguished between \textquote{\textit{Greater or Lesser Statistics}}, drawing the line between the \textit{traditional data analysis supported by mathematical statistics} and a broader concept of \textit{learning from data} \cite{chambers1993greater}.
In his view, data science reshapes statistics by enlarging its focus in the areas of data preparation and visualization. In particular, Chambers points to the opportunities offered by new types of data and new types of presentation, deeming such extended subject to be even larger than Tukey's data analysis.

In 2001, Breiman shed new light on the debate about the nature of data science. Namely, he distinguishes between two cultures of statistical modeling depending on the ultimate goal of the analysis of data: \textit{information} and \textit{prediction} \cite{breiman2001statistical}.
The former is related to the traditional statistics approach, where we assume that natural processes can be described by mathematical formulations, i.e. the \textit{models}. Thus, by fitting them to the data we can infer knowledge about the model's parameters and understand the dynamics of the phenomenon.
The latter, instead, is what we intend as data science. This prioritizes prediction over inference without caring much about the underlying mechanism that generates the data.

In the same year, Cleveland presented an \textquote{\textit{action plan to expand the technical areas of statistics focuses on
the data analyst}} \cite{cleveland2001data}. In particular, he posits a great deal of attention to the practical impact of the data analysis procedures, suggesting that  results in data science \textquote{\textit{should be judged by the extent to which they enable the analyst to learn from data}} (rather than their theoretical properties).
Also, he outlines a precise structure for future academic programs, where it stands out the equal balance given to \textit{multidisciplinarity investigations (25\%), model and methods for data (20\%), computing for data (15\%) and theory (20\%)}. 
Notably, he also leaves some space devoted explicitly to \textit{pedagogy (15\%)}. 

\citeA{donoho201750years} presents a thorough historical review of the evolution of data science as a discipline.
Specifically, he describes data science as an applied field growing out of traditional statistics, and he names it as \mbox{\textquote{Greater Data Science} (GDS)} in analogy with the Greater Statistics nomenclature adopted by Chambers.
In practice, Donoho proposes a taxonomy for the activities of the would-be such discipline of the future, based on the \textit{merging, relabeling and generalization} of the divisions proposed by Chambers and Cleveland:
\begin{displayquote}
The activities of GDS are classified into six divisions:
\begin{enumerate}
\setstretch{0.3}
    \item Data Gathering, Preparation, and Exploration\itemsep=1mm
    \item Data Representation and Transformation
    \item Computing with Data
    \item Data Modeling
    \item Data Visualization and Presentation
    \item Science about Data Science \label{gds6}
\end{enumerate}
% 1. Data Gathering, Preparation, and Exploration
% 2. Data Representation and Transformation
% 3. Computing with Data
% 4. Data Modeling
% 5. Data Visualization and Presentation
% 6. Science about Data Science
\end{displayquote}
Interestingly, Donoho also includes the study of data science (\gdsSix) as a part of the discipline itself, and he thereby encompasses activities devoted to:
% \textquote{\textit{identify commonly occurring analysis/processing workflows}} under this umbrella. Among these workflow,
\begin{displayquote}
identify commonly occurring analysis/processing workflows, for example, using data about their frequency of occurrence in some scholarly or business domain; when they \textbf{measure the effectiveness of standard workflows in terms of the human time, the computing resource, the analysis validity, or other performance metrics}, and when they uncover emergent phenomena in data analysis, for example, new patterns arising in data analysis workflows, or disturbing artifacts in published analysis results.
\end{displayquote}

More recently, \citeA{zhang2021data} inspected a clever approach to define the discipline of data science that fits exactly -- perhaps unwittingly -- in Donoho's \gdsSix. In fact, they adopt a data-driven strategy by analyzing the course descriptions of graduate degree programs in analytics, business  analytics,  and  data  science  offered  by  American universities.


Nonetheless, researchers and practitioners are yet to reach a complete agreement on its definition. 
Indeed, other schools of thought, especially in industry, suggest that data science should be considered as a separate body of knowledge due to its focus on digital and qualitative data \cite{silver2013need, priceonomics2015, dhar2013datascience}, and that statistics is only a marginal, nonessential part of it \cite{gelman2013statistics}.
The interested reader is referred to \citeA{donoho201750years} and \citeA{cao2017data} for a more extensive dissertation on the topic.

This thesis shares the vision of data science as an extension of applied statistics. In particular, we frame it as an applied science that studies how to combine domain expertise with suitable learning approaches and necessary computing resources to process real data for a tangible purpose.
Also, we devote particular care to the visualization of data.
Furthermore, we try to embrace \gdsSix \, when assessing the results to provide a comprehensive evaluation of our approaches from the application point of view.
\section{Learning theory}
\label{sec:learning}

By and large, data science can be summarized as the process of learning from data.
Multiple strategies are available to achieve that, differing for their fundamental assumptions, adopted methodologies, and ambit of applicability.
Although an exhaustive list of the possible approaches would be much longer, the different methods can be categorized into three main learning paradigms: supervised, unsupervised, and reinforcement learning.

The \textbf{Supervised Learning} (SL) paradigm is based on the concept of learning by examples, sometimes also referred to as ``learning with a teacher" \cite[Chapter 14]{friedman2009elements}.
Formally\footnote{\label{var_format} the variables in this section are represented according to the statistical conventions, namely: random variables are capitalized, while observations are expressed as the corresponding lowercase letters; also, bold symbols indicate vectors, meanwhile normal font stands for univariate quantities}, this can be expressed as the task of predicting a (set of) response/target\footnote{\label{var_naming} the terms \textit{predictors} and \textit{response} proper of the statistical community are used hereafter interchangeably with \textit{inputs} and \textit{output}, respectively, that are instead more used in the machine learning jargon} variable $Y$ based on a set of predictors/inputs\footnoteref{var_naming} $\boldsymbol X$.
In other words, the objective is to learn a mapping function, $f$, between the predictors and the response such that it is possible to forecast the value of $Y$ given the values of $\boldsymbol X$ up to some random noise $\epsilon$:
\begin{equation}
    Y = f \left(\boldsymbol X; \boldsymbol \theta \right) + \epsilon
\end{equation}
where $\boldsymbol \theta$ is a parameter vector that defines the precise form of the function $f$.
In practice, the model (\textit{student}) starts by guessing the association between some input data $\boldsymbol x$ and the corresponding response based on an initial configuration of the parameters. Then, the corresponding \textit{label}, $y$, is used to score the quality of the produced association according to a given performance measure (\textit{loss function}).
This comparison between the predicted answer 
% $\hat{y}$
and the right label
% $y$
is what furnishes the supervision, thus allowing the parameter updates to reduce the observed mismatch between the prediction and label.
% the prediction $\hat{y}$ and the real label $y$.
Finally, the whole procedure is iterated until the learned parameters generate a sufficiently satisfying mapping that allows the generalization to new data. 
The advantage of this approach is that the previous knowledge provided by the labels is leveraged to supervise the training, helping the model learn the right mapping between predictors and target.
For this reason, the SL paradigm is widely used in practice and has a long list of successes for many different learning tasks (e.g. spam filtering, fraud detection, image classification, stock price forecasting).
%, which contributed to the modern popularity of machine learning.
However, most of the data in a real-world scenario are produced without labels. This prevents the adoption of SL techniques to learn from such data unless undertaking a (probably costly) labeling/annotation phase before their analysis. 

Contrarily, \textbf{Unsupervised Learning} (UL), or ``learning without a teacher", addresses the task of learning when no labels are available. 
In this case, the goal is to directly infer properties of the input data $\boldsymbol X$ without the help of a supervisor \cite[Chapter 14]{friedman2009elements}, i.e. identifying hidden structures and commonalities in the data.
Cluster analysis is one of the most popular families of  unsupervised learning algorithms.
In this case, the problem is formulated as finding subgroups of observations (clusters) that can be considered similar according to a given measure of distance/similarity.
The training phase proceeds by assigning the data points to the clusters by minimizing an overall measure of internal compactness and/or separation between different groups.
Thus, UL is very convenient in many practical situations since it does not require labels (e.g. market basket analysis, anomaly detection, pattern recognition, dimension reduction), and it can be pursued as a goal per se -- i.e. to discover hidden patterns in data -- or as a pre-processing for subsequent elaborations -- e.g. learn a convenient representation for further analysis.
However, the absence of a reference target makes it difficult -- usually impossible -- to objectively measure the goodness of the results -- as opposed to the intuitive notion of score given by the loss function in SL \cite{von2012clustering}.
Rather, the evaluation typically resorts to heuristic arguments based on the interpretability of the results for a given use case, which makes the whole process somewhat arbitrary.


Alternative learning paradigms have been proposed at the intersection between supervised and unsupervised learning to access the benefits of both strategies and mitigate their limitations.
For instance, \textit{semi-supervised learning} adopts a mixed training strategy where only a small fraction of the data are labeled. Nonetheless, such approaches perform significantly better than purely unsupervised methods.
Another example is represented by \textit{weakly supervised learning}. In this case, the collected labels may be scarce in number and/or quality (i.e. inaccurate). Again, this implies considerable improvements in performance at lower annotation costs with respect to supervised strategies. 
An emerging alternative that is showing great potential is \textit{self-supervised learning}. This approach tries to leverage large amounts of unlabeled data to learn representations to reuse for different purposes. In particular, a model is first pre-trained on a so-called pretext task, i.e. a learning objective other than the desired target but propaedeutic for its learning, for which labels can be automatically/easily retrieved. Then, the learned representation is reused to fine-tune the model on the downstream task of interest.

A completely different approach is represented by \textbf{Reinforcement Learning} (RL). In its simplest formulation, the learning task is formalized in terms of an agent interacting with an evolving environment, and the goal is to learn a policy -- i.e. a mapping between the states of the environment and the corresponding actions the agent may undertake in those states.
These actions generate a reward for the agent, and an optimal mapping must be sought to maximize this notion of reward over time.
Drawing a comparison with the above paradigms, in this case the inputs correspond to the environment settings at a given time $t$, and the agent has to output an action $a_t$.
Like in UL, no explicit labels are provided to the agent to learn from.
Instead, the reward signal is used as an indirect measure of ranking between the goodness of different actions for a given state. This is conceptually different from the label in SL, which instead represents the right prediction for a given input \cite[Chapter 1]{sutton2018reinforcement}.
The RL framework is particularly appealing for its resemblance with the way we humans learn.
Indeed, we interact with an environment without an explicit supervision, and we learn to associate actions to given situations.
For this reason, RL is also very adopted in practice for an wide range of applications ranging from autonomous driving to playing games.

In this thesis, we adopt different learning paradigms based on the nature of the data and the corresponding use cases.
In particular, \cref{partI} exploits convolutional neural networks for classifying pixels into signal and background classes.
\Cref{partII} explores the domains of self-supervised and unsupervised learning instead. Specifically, a language model is first used to get a convenient numeric representation for textual data (\cref{sec:vectorization}), and a K-means algorithm is then applied to get clusters of similar error messages (\cref{sec:clustering}).