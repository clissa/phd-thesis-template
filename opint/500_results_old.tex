\chapter{Results} 
\label{ch:opint-results}

% Although manually inspecting 30 potential error categories would imply a significant effort for the human operators, we find that this order of magnitude works fine in practice.
% Indeed, most clusters present a low number of messages (a few hundreds or thousands) and can be discarded at first as they make up only a minimal part of the total failures. 
% Furthermore, some of the clusters collect transient problems that can be ignored based on their time trends (see \cref{sec:viz}).
% Finally, bigger clusters are mostly linked with evident problems, for which reporting has already occurred. Thus, also this suggestion can be filtered out through a cross-check with the open issues reported in GGUS.
% As a result, a manageable number of clusters and corresponding messages are typically provided, and shifters are likely required to focus on error categories of medium sizes.

% The pipeline consists of 
% % can be summarised into 
% two main steps: \textit{i)} vectorization and \textit{ii)} clustering.
% In the {\emph{vectorization step}} we concatenate the raw error string with source and destination hostnames and we use a word2vec model that learns how to map all that information to a vectorial space of a given size, where similar errors are expected to be close together. This is to transform the textual information into a convenient numeric representation and serves as preprocessing for the next steps. 
% A {\emph{clustering algorithm}} is then applied to group related errors: 
% % In practice, the idea is to 
% we pre-train a word2vec model on a big dataset 
% -- possibly updating it once in a while -- and then 
% and run a \mbox{K-Means++} algorithm \cite{kmeans} online during the monitoring shifts.  
% % 



% The first three columns provide numeric summaries: \textit{i)} the cluster size, \textit{ii)} the number of unique strings within the cluster, and \textit{iii)} the number of unique patterns: unique strings after the removal of parametric parts like paths, IP addresses, URLs and so on. The model learns to \textit{abstract} message parameters and to group strings that are similar except for the parametric parts. As a result, the initial amount of errors is reduced to a number of patterns which is lower by several orders of magnitude. 
% The core part of this visualization is then represented by the \textit{Top 3} section, where the most frequent triplets of pattern, source and destination sites are reported in descending order, together with their multiplicity and the percentage over the cluster size.
% There we extract several insights, for example whether a pattern is responsible for a large number of failures or if it accounts for a conspicuous fraction of the cluster. In addition, one can investigate the contribution of source/destination site pairs, 
% % Also, one could look at whether errors are concentrated in a specific source/destination site, 
% as in \cref{fig:cluster0} where Site-4 clearly seems to have a problem as destination. 
% %
% Another useful piece of information is given by the cluster's time evolution plot (shown in \cref{fig:timeplot_cluster0}) that can give an immediate indication of whether the problem  is transient or not. 



\newcommand{\specialcell}[2][c]{\begin{tabular}[#1]{@{}c@{}}#2\end{tabular}}
\begin{table}[htb]
\centering
% \resizebox{\textwidth}{!}{
\begin{tabular}{cccccccc}
\toprule
\textbf{N. Clusters} &  \textbf{ASW} &  \textbf{WSSE} &  \textbf{\specialcell{Perfect\\Match}} &  \textbf{\specialcell{Fuzzy\\Match}} &  \textbf{\specialcell{Partial\\Match}} & \textbf{\specialcell{False \\ Positives}} & \textbf{\specialcell{False \\Negatives}} \\
\toprule
     &       &        &     &    &    &    &   \\[-0.25cm]
  15 &  0.89 &  17107 &   7 &  3 &  2 &  3 &  1 \\[0.2cm]
\bottomrule  
\end{tabular}
% }
\caption{Summary of pre-validation results.}
\label{tab:crosscheck}
\end{table}

The process described above can be fully automated after tuning model hyper parameters based on metrics such as Average Silhouette Width score
% (ASW)
and Within-cluster Sum of Squared (Euclidean) distances, that measure how compact and separated the clusters are.
However, this is not directly related to the meaning of the messages being grouped. Hence, this is to be intended as a proxy of a correct model behaviour rather than a real performance metric.
For this reason, 
We have conducted an extensive testing as pre-validation comparing the clusters obtained with this approach against GGUS tickets, showing a reasonable overlap between suggested and reported issues. 
In particular, for the analysis above we considered issues reported in a skewed time window of 17 days (1st to 18th January) so to include both known issues and the ones possibly spotted with some delay.
Results of this cross-check are reported in Table \ref{tab:crosscheck}.

The model found 15 clusters reaching ASW and WSSE of 0.89 and 17107, respectively. The results show a good agreement, with 7 perfect matches (both reported message and affected site), 3 fuzzy matches (i.e., the cluster had evident connection with more than one ticket) and 2 partial ones (either message or site). Besides that, 3 clusters highlighted issues that were not reported on GGUS. In some cases, posterior checks showed hints for real problems that went undetected or unreported by experts. %%Finally, 1 GGUS ticket was not discovered by the algorithm. 
% 
Although it makes sense to cross-check clustering results with tickets, this comparison has some drawbacks. In particular, the procedure is very sensitive to the choice of the time window.
% (some issues could have no match because already reported days before or because yet to be detected on the day of the analysis)
It requires a manual check of the ticket information and the cluster content, which makes the comparison lengthy and not scalable.
% For this reason, we are planning
% 
% Therefore we intend to build a reference dataset where to store labels for error categories, root causes, priority and solving actions. In this way, we will have a real measure of performance while easing the comparison of alternative algorithms and making the investigation of novel techniques sustainable.
% More importantly, collecting such information would allow leveraging tools available in NLP literature about Question Answering (QA) or Named Entity Recognition (NER) to address the key problem related to transfer failures, i.e., understanding the root causes and suggesting solving actions for the problems.