\section{Quantitative assessment: GGUS tickets}


The drawback of unsupervised techniques lies in the inherent difficulty of the evaluation phase, as no ground truth is available for comparison.
In order to overcome this limitation,
% % and to avoid demanding the whole performance assessment to manual inspection and interpretation of the clustering results, we resort to the comparison 
% we have conducted extensive testing as pre-validation comparing the clusters obtained with our approach against GGUS tickets.
% In order to overcome the difficulty in measuring the goodness of the produced clusters, 
we have conducted extensive testing using GGUS reported incidents as a benchmark. 
In this way, we provide a quantitative assessment of the pipeline performances and a more direct measure of its potential impact when applied in practice.
In particular, we consider GGUS incidents referred to the ATLAS collaboration reported in a skewed time window of 17 days (01-01 to 01-18) around the day of the analysis, for a total of 20 tickets (in fact 30, but 9 referring to analysis jobs and 1 test ticket).
In this way, previously known issues are included, and we also allow for a reasonable time delay in the reporting.
The results of such comparison are reported in Table \ref{tab:cross-check}.
\newcommand{\specialcell}[2][c]{\begin{tabular}[#1]{@{}c@{}}#2\end{tabular}}
\begin{table}%[htb]
\centering
\resizebox{\textwidth}{!}{
\begin{tabular}{cccccccc}
\toprule
\textbf{N. Clusters} &  \textbf{ASW} &  \textbf{WSSE} &  \textbf{\specialcell{Perfect\\Match}} &  \textbf{\specialcell{Fuzzy\\Match}} &  \textbf{\specialcell{Partial\\Match}} & \textbf{\specialcell{False \\ Positives}} & \textbf{\specialcell{False \\Negatives}} \\
\toprule
     &       &        &     &    &    &    &   \\[-0.25cm]
  15 &  0.89 &  17107 &   7 &  3 &  2 &  3 &  1 \\[0.2cm]
\bottomrule  
\end{tabular}
}
\caption{Summary of pre-validation results.}
\label{tab:cross-check}
\end{table}

Overall, a good level of agreement is observed between the 15 discovered clusters and the 20 tickets.
Specifically, the seven \textit{perfect matches} indicate cases whereby the reported message and the affected site coincide with the ones highlighted by the clusters.
The three \textit{fuzzy matches}, instead, refer to occasions whereby the agreement is less obvious, meaning that the cluster has evident connections with more than one ticket.
Similarly, the two \textit{partial matches} describe cases whereby either the message or the site coincide. 
Besides that, 3 clusters highlighted issues that were not reported on GGUS.
In some cases, posterior checks showed hints for real problems that went undetected or unreported by experts. 
Finally, 1 GGUS ticket was not separated and highlighted by the algorithm. 